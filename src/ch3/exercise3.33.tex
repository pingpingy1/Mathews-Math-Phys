%%%%%%%%%%%%%%%%%%%%%%%
\item
%%%%%%%%%%%%%%%%%%%%%%%

As $e^{\alpha z}$ is analytic everywhere, $\Gamma(z) e^{\alpha z}$ has poles precisely at the nonpositive integers.
\[
    I := \oint_{\abs{z} = \frac{5}{2}} dz \Gamma(z) e^{\alpha z}
    = 2i\pi \left(
        \res{z = 0}{\Gamma(z) e^{\alpha z}}
        + \res{z = -1}{\Gamma(z) e^{\alpha z}}
        + \res{z = -2}{\Gamma(z) e^{\alpha z}}
    \right)
\]

To calculate these residues, we wish to transform the equation
\[
    \res{z = -n}{\Gamma(z) e^{\alpha z}}
    = \lim_{z \rightarrow -n} (z + n) \Gamma(z) e^{\alpha z}
\]
in such a way to make the argument of the gamma function positive, where $n$ is a nonnegative integer.
\begin{align*}
    \res{z = -n}{\Gamma(z) e^{\alpha z}}
    &= \lim_{z \rightarrow -n} (z + n) \Gamma(z) e^{\alpha z} \\
    &= \lim_{z \rightarrow -n} (z + n) \cdot \frac{z (z + 1) \cdots (z + n)}{z (z + 1) \cdots (z + n)} \Gamma(z) e^{\alpha z} \\
    &= \lim_{z \rightarrow -n} \frac{\Gamma(z + n + 1) e^{\alpha z}}{z (z + 1) \cdots (z + n - 1)} \\
    &= \frac{\Gamma(0) e^{-n \alpha}}{(-n) (-n + 1) \cdots (-1)} \\
    &= \frac{{\left( -e^\alpha \right)}^n}{n!}
\end{align*}
\[
    \therefore I
    = 2i\pi \left( 1 - e^\alpha + \frac{e^{2\alpha}}{2} \right)
\]