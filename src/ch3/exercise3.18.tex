%%%%%%%%%%%%%%%%%%%%%%%
\item
%%%%%%%%%%%%%%%%%%%%%%%

\begin{figure}[h]
	\centering
	\begin{tikzpicture}
		% Axes
		\draw[axis] (-2cm, 0cm) -- (2cm, 0cm) node[right] {$\Re$};
		\draw[axis] (0cm, -2cm) -- (0cm, 2cm) node[right] {$\Im$};

		% Branch cut
		\node[pole] at (0cm, 0cm) {};
		\draw[branch cut] (0cm, 0cm) -- (1.9cm, 0cm);

		% Poles
        	\node[pole] (A) at (0cm, -0.7cm) {}; \node[right] at (A.east) {$-i$};
        	\node[pole] (B) at (0cm,  0.7cm) {}; \node[right] at (B.east) {$i$};

		% Contour
		\def\R{1.8cm}
		\def\eps{0.4cm}
        	%%% TODO: Remove these magical constants
		\draw[contour] (0.7071*\eps, 0.7071*\eps) -- (1.7776cm, 0.7071*\eps) node[below] {$R$};
		\draw[contour] (1.7776cm, 0.7071*\eps) arc (9.0406:350.9093:\R);
		\draw[contour] (1.7776cm, -0.7071*\eps) -- (0.7071*\eps, -0.7071*\eps);
		\draw[contour] (0.7071*\eps, -0.7071*\eps) arc (315:45:\eps);
	\end{tikzpicture}
	\caption{Contour for \refprob{}~3--18.}%
	\label{fig:problem3-18}
\end{figure}

Recalling the exercise in the main text regarding \refeq{3}{37},
we could infer that a direct contour integration of the integrand along some contour analogous to \reffig~3--3
in the textbook would lead to a loss in the exponent of the logarithm (or one could just do it and observe).

Thus, let $f(z) := \frac{{\left( \ln z \right)}^3}{1 + z^2}$,
where we place the branch cut on the real axis and $\arg z = 0$ just above the real axis.
As such, the poles of $f(z)$ have arguments $\frac{\pi}{2}$ and $\frac{3\pi}{2}$.
Consider the contour shown in \reffig~\ref{fig:problem3-18}.
\begin{align*}
	\oint_C dz f(z)
	&= 2i\pi \left( \res{z=i} f(z) + \res{z=-i} f(z) \right) \\
	&= 2i\pi \left( \evalAt{\frac{{\left( \ln z \right)}^3}{2z}}{z=i} + \evalAt{\frac{{\left( \ln z \right)}^3}{2z}}{z=-i} \right) \\
	&= \frac{13}{4}i\pi
\end{align*}
We also have
\begin{align*}
    \oint_C dz f(z)
    &= \int_\epsilon^R dx \frac{{\left( \ln x \right)}^3}{1 +x^2}
     + \int_0^{2\pi} d\theta iRe^{i\theta} \frac{{\left( \ln R + i\theta\right)}^3}{R^2 e^{2i\theta} + 1} \\
    &\quad + \int_R^\epsilon dx \frac{{\left( \ln x + 2i\pi \right)}^3}{1 + x^2}
     + \int_{2\pi}^0 d\theta i\epsilon e^{i\theta} \frac{{\left(\ln\epsilon + i\theta\right)}^3}{\epsilon^2 e^{2i\theta} + 1} \\
    &\xrightarrow[\epsilon \rightarrow 0]{R \rightarrow \infty}
        -6i\pi \int_0^\infty dx \frac{{\left( \ln x \right)}^2}{1 + x^2}
	+ 12\pi^2 \int_0^\infty dx \frac{\ln x}{1 + x^2}
	+ 4i\pi^4.
\end{align*}

This same procedure can be used to show that
\[
    \int_0^\infty dx \frac{\ln x}{1 + x^2} = 0,
\]
which is left as an exercise to the reader\footnote{
Just kidding!
I will get back to this, but if you could, please consider discussing on \url{https://github.com/pingpingy1/Mathews-MathPhys-Sol/issues/2}.
}.

\[
	\therefore \int_0^\infty dx \frac{{\ln x}^2}{1 + x^2} = \frac{\pi^3}{8}
\]
