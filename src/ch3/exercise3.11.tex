%%%%%%%%%%%%%%%%%%%%%%%
\item
%%%%%%%%%%%%%%%%%%%%%%%

\begin{figure}[h]
	\centering
	\begin{tikzpicture}
		% Axes
		\draw[axis] (-2cm, 0cm) -- (2cm, 0cm) node[right] {$\Re$};
		\draw[axis] (0cm, -1cm) -- (0cm, 3cm) node[right] {$\Im$};

		% Poles
		\def\w{0.7cm}
		\def\e{0.2cm}
		\node[pole] (A) at ( \w, \e) {}; \node[above] at (A.north) {$ \omega_0 + i\epsilon$};
		\node[pole] (B) at (-\w, \e) {}; \node[above] at (B.north) {$-\omega_0 + i\epsilon$};

		% Contour
		\def\R{1.8cm}
		\draw[contour] (-\R, 0cm) node[below] {$-R$} -- (\R, 0cm) node[below] {$R$};
		\draw[contour] ( \R, 0cm) arc (0:180:\R);
		\node[blue, right, yshift=0.2cm] at (0cm, \R) {$iR$};
	\end{tikzpicture}
	\caption{Contour for \refprob{}~3--11.}%
	\label{fig:problem3-11}
\end{figure}

We shall perform a contour integral of $f(\omega) := \frac{e^{i\omega t}}{\omega^2 - \omega_0^2}$ along \reffig{}~\ref{fig:problem3-11}.
\[
	\oint_C d\omega f(\omega)
	= 2i\pi \left( \res{\omega = -\omega_0} f(\omega) + \res{\omega = \omega_0} f(\omega) \right)
	= -\frac{2\pi}{\omega_0} \sin \omega_0 t
\]
We also have
\begin{align*}
	\oint_C d\omega f(\omega)
	 & = \int_{-R}^R d\omega \frac{e^{i\omega t}}{\omega^2 - \omega_0^2}
	+ \int_0^\pi d\theta iRe^{i\theta} \frac{e^{iRte^{i\theta}}}{R^2 e^{2i\theta} - \omega_0^2}                     \\
	 & \xrightarrow{R \rightarrow \infty} \int_{-\infty}^\infty d\omega \frac{e^{i\omega t}}{\omega^2 - \omega_0^2}
\end{align*}
\[
	\therefore \int_{-\infty}^{\infty} d\omega \frac{e^{i\omega t}}{\omega^2 - \omega_0^2}
	= -\frac{2\pi}{\omega_0} \sin \omega_0 t
\]

(Note: As there are poles on the real axis, this integral technically diverges without the ``slightly above the axis'' condition.
If this integral were to have physical meaning, then this assumption must be physically explained, for example using causality arguments.
Another way around this is to use the ``Cauchy principal value'' of the integral, whose formula is shown in \refeq{A}{17}.)