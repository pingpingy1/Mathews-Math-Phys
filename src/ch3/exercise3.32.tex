%%%%%%%%%%%%%%%%%%%%%%%
\item
%%%%%%%%%%%%%%%%%%%%%%%

Let $\theta$ denote the angle away from the dotted axis in the figure.
We need not consider the azimuthal angle around it as the system is cylindrically symmetric.
The intensity of neutrons as emitted by the source into a range $[\theta, \theta + d\theta]$ is given by
\[
    dI_{\text{source}}
    = I_0 \frac{d\Omega}{4\pi}
    = \frac{I_0}{2} \sin \theta d\theta
\]
where $I_0$ is the total intensity of neutrons.
Such neutrons travel a distance of $T \sec \theta$ through the absorber,
so the intensity of neutrons that arrive at the film is given by
\[
    dI_{\text{film}}
    = \frac{I_0}{2} e^{-\frac{T}{\lambda} \sec \theta} \sin \theta d\theta.
\]
These neutrons travel through the film, again, a distance proportional to $\sec \theta$.
Thus, one can introduce a proportionality constant $C$ to obtain
\[
    dA
    = C e^{-\frac{T}{\lambda} \sec \theta} \sin \theta d\theta \cdot \sec \theta
    = C e^{-\frac{T}{\lambda} \sec \theta} \tan \theta d\theta.
\]
Therefore, integrating over the film, we obtain the following formula for the total activity:
\begin{align*}
    A(\lambda)
    &= \int_0^{\tan^{-1} \left( \frac{b}{a} \right)} d\theta C e^{-\frac{T}{\lambda} \sec \theta} \tan \theta \\
    &= C \int_1^{\frac{\sqrt{a^2 + b^2}}{a}} du \frac{\exp \left( -\frac{T}{\lambda} u \right)}{u}
       && (u := \sec \theta) \\
    &= C \int_{\frac{T}{\lambda}}^{\frac{T}{\lambda} \frac{\sqrt{a^2 + b^2}}{a}}
         dv \frac{\exp(-v)}{v}
       && (v := \frac{T}{\lambda}u) \\
    &= C \left(
            \left( -\Ei\left( -\frac{T}{\lambda}\frac{\sqrt{a^2 + b^2}}{a} \right) \right)
            - \left( -\Ei\left( -\frac{T}{\lambda} \right) \right)
       \right).
\end{align*}
The activity without the absorber is given by
\[
    A_0
    = \lim_{T \rightarrow 0} A(\lambda)
    = \int_0^{\tan^{-1} \left( \frac{b}{a} \right)} d\theta C \tan \theta
    = C \ln \left( \frac{\sqrt{a^2 + b^2}}{a} \right).
\]
Hence, we obtain
\[
    \frac{A}{A_0}
    = \frac{
        \left( -\Ei\left( -\frac{T}{\lambda}\frac{\sqrt{a^2 + b^2}}{a} \right) \right)
        - \left( -\Ei\left( -\frac{T}{\lambda} \right) \right)
    }{\ln \left( \frac{\sqrt{a^2 + b^2}}{a} \right)}
\]
which, given the numerical constants, is an equation that the almighty WolframAlpha could solve!
Plugging in the given constants and assuming $\lambda$ is in units of centimeters, we get
\[
    0.25 = \frac{
        \left( -\Ei\left( -\frac{\sqrt{2}}{\lambda} \right) \right)
        - \left( -\Ei\left( -\inv{\lambda} \right) \right)
    }{\ln\left( \sqrt{2} \right)}.
\]
\[
    \therefore \lambda \approx 0.856\, \mathrm{cm}
\]