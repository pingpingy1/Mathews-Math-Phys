%%%%%%%%%%%%%%%%%%%%%%%
\item
%%%%%%%%%%%%%%%%%%%%%%%

\begin{figure}[h]
	\centering
	\begin{tikzpicture}
		% Axes
		\draw[axis] (-2cm, 0cm) -- (2cm, 0cm) node[right] {$\Re$};
		\draw[axis] (0cm, -2cm) -- (0cm, 2cm) node[right] {$\Im$};

		% Branch cut
		\node[pole] at (0cm, 0cm) {};
		\draw[branch cut] (0cm, 0cm) -- (1.9cm, 0cm);

		% Poles
        \node[pole] (A) at (-0.7cm, 0cm) {}; \node[below] at (A.south) {$-1$};

		% Contour
		\def\R{1.8cm}
		\def\eps{0.4cm}
        	%%% TODO: Remove these magical constants
		\draw[contour] (0.7071*\eps, 0.7071*\eps) -- (1.7776cm, 0.7071*\eps) node[below] {$R$};
		\draw[contour] (1.7776cm, 0.7071*\eps) arc (9.0406:350.9093:\R);
		\draw[contour] (1.7776cm, -0.7071*\eps) -- (0.7071*\eps, -0.7071*\eps);
		\draw[contour] (0.7071*\eps, -0.7071*\eps) arc (315:45:\eps);
	\end{tikzpicture}
	\caption{Contour for \refprob{}~3--24.}%
	\label{fig:problem3-24}
\end{figure}

Let $f(z) := \frac{{\left( \ln z \right)}^2}{{\left( z + 1 \right)}^2}$,
where we place the branch cut on the real axis and $\arg z = 0$ just above the real axis.
(Recall the exercise in the main text regarding \refeq{3}{37};
we need an extra $\ln z$ for this procedure!)
As such, the pole of $f(z)$ has argument $\pi$.
Consider the contour shown in \reffig~\ref{fig:problem3-24}.
\begin{align*}
	\oint_C dz f(z)
	&= 2i\pi \res{z=-1} f(z) \\
	&= 2i\pi \cdot \evalAt{\frac{d^2}{dz^2} {\left( \ln z \right)}^2}{z = e^{i\pi}} \\
	&= 4\pi^2
\end{align*}
We also have
\begin{align*} % TODO: Make this prettier!!
    \oint_C dz f(z)
    &= \int_\epsilon^R dx \frac{{\left( \ln x \right)}^2}{{\left( x + 1 \right)}^2}
      + \int_0^{2\pi} d\theta iRe^{i\theta} \frac{{\left( \ln R + i\theta\right)}^2}{{\left( Re^{i\theta} + 1 \right)}^2} \\
    & + \int_R^\epsilon dx \frac{{\left( \ln x + 2i\pi \right)}^2}{{\left( x + 1 \right)}^2}
      + \int_{2\pi}^0 d\theta i\epsilon e^{i\theta} \frac{{\left(\ln\epsilon + i\theta\right)}^2}{{\left( \epsilon e^{i\theta} + 1 \right)}^2} \\
    &\xrightarrow[\epsilon \rightarrow 0]{R \rightarrow \infty} \\
    &   \int_0^\infty dx \frac{{\left( \ln x \right)}^2}{{\left( x + 1 \right)}^2} + 0 \\
    & + \left( - \int_0^\infty dx \frac{{\left( \ln x \right)}^2}{{\left( x + 1 \right)}^2}
      - 4i\pi \int_0^\infty dx \frac{\ln x}{{\left( x + 1 \right)}^2}
      + 4\pi^2 \int_0^\infty \frac{dx}{{\left( x + 1 \right)}^2} \right) \\
    & + 0 \\
    &= -4i\pi \int_0^\infty dx \frac{\ln x}{{\left( x + 1 \right)}^2} + 4\pi^2.
\end{align*}

\[
	\therefore \int_0^\infty dx \frac{\ln x}{{\left( x + 1 \right)}^2} = 0
\]
