%%%%%%%%%%%%%%%%%%%%%%%
\item
%%%%%%%%%%%%%%%%%%%%%%%

First, consider a single line charge $\lambda$ at the origin.
We want to express the potential $V(x, y)$ as the imaginary component of some complex function
\[
    F(z) = U(x, y) + iV(x, y).
\]
Rotational symmetry implies that $V$ is solely a function of $r := \abs{z}$;
consequently, the streamlines formed by $U = \const$ must be radial.
Moreover, $U$ must be multivalued since the integral
\[
    \oint_{r = \const} dz (\nabla V)_n  = \Delta U
\]
is nonzero (unless $V$ itself is zero).
In other words, following a circular contour around the origin increments $U$ by a fixed amount.
``Therefore,'' the only analytic function that satisfies these conditions is
\[
    F(z)
    := iC \ln z
    = C \left( -\theta + i\ln r \right) \quad (z = re^{i\theta}).
\]
(Note: I really hoped that I could come up with an air-tight argument for the logarithm to appear here;
if you have any ideas other than direct calculation of the potential, please let me know!)

Integration around the origin then should represent the total linear charge density enclosed:
\[
    \frac{\lambda}{\epsilon_0}
    = \oint_{r = \const} dz (\nabla V)_n 
    = \Delta U
    = -2\pi C.
\]
\[
    \Rightarrow C = -\frac{\lambda}{2\pi\epsilon_0}
    \Rightarrow F(z) = -\frac{i\lambda}{2\pi\epsilon_0} \ln z
\]
Confusingly enough, this problem in particular requires that the \emph{real} part be the potential.
Thus, we divide by $i$ and use
\[
    F(z) = -\frac{\lambda}{2\pi\epsilon_0} \ln z.
\]

For the 2D-equivalent of the electric dipole as described in the problem,
we simply superpose the functions for each of the line charge.
\begin{align*}
    \therefore W(z)
    &= \lim_{d \rightarrow 0^+} \left(
        -\frac{p/2d}{2\pi\epsilon_0} \ln (z - id)
        + \left( -\frac{-p/2d}{2\pi\epsilon_0} \ln (z + id) \right)
    \right) \\
    &= \frac{ip}{2\pi\epsilon_0} \lim_{d \rightarrow 0^+} \frac{
        \ln (z + id) - \ln (z - id)
    }{2id} \\
    &= \frac{ip}{2\pi\epsilon_0 z}
\end{align*}
Notice that the electric potential is given by
\[
    \Re \left\{ W(z) \right\}
    = \frac{p}{2\pi\epsilon_0} \frac{y}{x^2 + y^2}.
\]