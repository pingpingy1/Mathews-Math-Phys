%%%%%%%%%%%%%%%%%%%%%%%
\item
%%%%%%%%%%%%%%%%%%%%%%%

\begin{figure}[ht]
    \centering
    \begin{subfigure}{0.4\textwidth}
        \centering
        \begin{tikzpicture}
            % Region Shading
            \pattern[pattern={Lines[angle=45,distance=0.1cm]}, pattern color=yellow]
                (-1.8cm, 1.8cm) -- (-1.8cm, 0cm) -- (1.8cm, 0cm) -- (1.8cm, 1.8cm) -- cycle;

            % Axes
            \draw[axis] (-2cm, 0cm) -- (2cm, 0cm) node[right] {$\Re$};
            \draw[axis] (0cm, -2cm) -- (0cm, 2cm) node[right] {$\Im$};
            \node (name) at (1.5cm, 1.5cm) {$z$};
            \draw (name.north west) |- (name.south east);
    
            % Poles
            \node[pole] (A) at (-1.3cm, 0cm) {}; \node[below] at (A.south) {$a_1$};
            \node[pole] (B) at ( 1.3cm, 0cm) {}; \node[below] at (B.south) {$a_2$};

        \end{tikzpicture}
        \caption{}%
        \label{subfig:problem5-7:original}
    \end{subfigure}
    \hfill
    \begin{subfigure}{0.4\textwidth}
        \centering
        \begin{tikzpicture}
            % Region Shading
            \pattern[pattern={Lines[angle=45,distance=0.1cm]}, pattern color=yellow]
                (-1.8cm,  1.8cm) -- (-1.8cm, -0.6cm) -- (-1cm, 0.5cm)
                -- (1cm, 0.9cm) -- (1.8cm, -1.6cm) -- (1.8cm,  1.8cm) -- cycle;
            
            % Axes
            \draw[axis] (-2cm, 0cm) -- (2cm, 0cm) node[right] {$\Re$};
            \draw[axis] (0cm, -2cm) -- (0cm, 2cm) node[right] {$\Im$};
            \node (name) at (1.5cm, 1.5cm) {$\zeta$};
            \draw (name.north west) |- (name.south east);
    
            % Poles & Boundary
            \draw[gray, dashed] (-1cm, 0.5cm) -- (-0.0545cm,  1.8cm);
            \draw[gray, dashed] ( 1cm, 0.9cm) -- (    1.8cm, 1.06cm);
            \draw[charged strip]
                (-1.8cm, -0.6cm)
                -- ( -1cm,  0.5cm) node[pole] (A) {}
                -- (  1cm,  0.9cm) node[pole] (B) {}
                -- (1.8cm, -1.6cm);
            
            \node[left] at (A.west) {$z = a_1$};
            \node[below, xshift=-0.4cm] at (B.south) {$z = a_2$};

            \draw[->] (-0.7818cm, 0.8cm)
                arc (53.973:11.310:0.3709cm) node[midway, right, yshift=0.15cm] {$\pi s_1$};
            \draw[->] (1.3cm, 0.96cm)
                arc (11.310:-72.255:0.3059cm) node[midway, right] {$\pi s_2$};

            % Derivative Infinitesimal Vectors
            \draw[thick, magenta, ->] (-1.6cm,  -0.325cm)
                -- node[midway, right, xshift=-0.1cm, yshift=-0.1cm]
                    {$d\mathbb{s} \propto Ae^{i\pi (s_1 + s_2)}$}
                ( -1.3cm,  0.0875cm);
            \draw[thick, magenta, ->] (   0cm,     0.7cm)
                -- node[midway, above, yshift=0.3cm] {$d\mathbb{s} \propto Ae^{i\pi s_2}$}
                (  0.4cm,    0.78cm);
            \draw[thick, magenta, ->] ( 1.5cm, -0.6625cm)
                -- node[midway, left] {$d\mathbb{s} \propto A$}
                (1.625cm, -1.0625cm);

        \end{tikzpicture}
        \caption{}%
        \label{subfig:problem5-7:general}
    \end{subfigure}
    \hfill
    \begin{subfigure}{0.5\textwidth}
        \centering
        \begin{tikzpicture}
            % Region Shading
            \pattern[pattern={Lines[angle=45,distance=0.1cm]}, pattern color=yellow]
                (-1cm,  1.8cm) -- (-1cm, 0cm) -- (1cm, 0cm) -- (1cm, 1.8cm) -- cycle;
            
            % Axes
            \draw[axis] (-2cm, 0cm) -- (2cm, 0cm) node[right] {$\Re$};
            \draw[axis] (0cm, -2cm) -- (0cm, 2cm) node[right] {$\Im$};
            \node (name) at (1.5cm, 1.5cm) {$\zeta$};
            \draw (name.north west) |- (name.south east);
    
            % Poles & Boundary
            \draw[charged strip]
                (-1cm, 1.8cm)
                -- (-1cm,   0cm) node[pole] (A) {}
                -- ( 1cm,   0cm) node[pole] (B) {}
                -- ( 1cm, 1.8cm);
            
            \node[left, yshift=0.2cm] at (A.west) {$z = -1$};
            \node[below] at (B.south) {$z =  1$};

            \draw[gray, dashed] (-1cm, 0cm) -- (-1cm, -1cm);

            \draw[->] ( -1cm, -0.4cm)
                arc (-90:0:0.4cm) node[midway, right] {$-\halfpi$};
            \draw[->] (1.4cm,    0cm)
                arc (0:90:0.4cm) node[midway, right, yshift=0.2cm] {$-\halfpi$};

        \end{tikzpicture}
        \caption{}%
        \label{subfig:problem5-7:arcsin}
    \end{subfigure}
    \caption{Results of conformal mappings used in \refprob~5--7.}%
    \label{fig:problem5-7}
\end{figure}

\begin{enumerate}[wide, labelindent = 0pt, label = (\alph*)]
\item
The proof has basically been outlined in \reffig~\ref{subfig:problem5-7:general}.
This transformation, which can be searched for as%
\urlfoot{https://en.wikipedia.org/wiki/Schwarz\%E2\%80\%93Christoffel_mapping}{%
the Schwarz-Christoffel mapping}%
\footnote{The lack of `t' in `Schwarz' here hurts my brain.
Which is right --- the textbook or Wikipedia?
Probably Wikipedia; it claims that this is Schwarz in the Cauchy-Schwarz inequality.},
uses the key insight that the phase of $z^s$ takes a jump discontinuity
on the real axis at the origin.
Quantitatively, it is $\pi s$ for $x < 0$ and $0$ for $x > 0$ (slightly above the real axis).
Therefore, the slope of the mapping, which is basically the phase of $\frac{d\zeta}{dz}$,
changes discontinuously at $z = a_1, a_2$.
How discontinuous? Well, it decreases by $\pi s_1$ at $z = a_1$ and $\pi s_2$ at $z = a_2$,
just as the problem claimed!

\item
For simplicity, suppose the vertices correspond to $z = \pm 1$.
As shown in \reffig~\ref{subfig:problem5-7:arcsin}, this leads to phase shifts of $-\halfpi$.
\begin{align*}
    \Rightarrow \zeta(z)
    &= A\int \frac{dz}{{(z + 1)}^{1/2} {(z - 1)}^{1/2}} \\
    &= A\int \frac{dz}{{\left( z^2 - 1 \right)}^{1/2}} \\
    &= A\cosh^{-1} z + B
\end{align*}
The constants $A, B$ are fixed by the conditions
\[
    \zeta(z=-1) = i\pi A + B = -a,\; \zeta(z=1) = B = a.
\]
\[
    \Rightarrow A = \frac{2ia}{\pi},\; B = a
\]
\begin{align*}
    \therefore \zeta(z)
    &= \frac{2ia}{\pi} \cosh^{-1} z + a \\
    &= \frac{2a}{\pi} \cos^{-1} z + a \\
    &= \frac{2a}{\pi} \left( \cos^{-1} z + \halfpi \right) \\
    &= \frac{2a}{\pi} \sin^{-1} z
\end{align*}
Thus, we have recovered the arcsine transformation as shown in \reffig~5--3 from the main text.

\end{enumerate}
