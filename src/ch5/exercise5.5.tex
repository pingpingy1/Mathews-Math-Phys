%%%%%%%%%%%%%%%%%%%%%%%
\item
%%%%%%%%%%%%%%%%%%%%%%%

\begin{figure}[ht]
    \centering
    \begin{subfigure}{0.4\textwidth}
        \centering
        \begin{tikzpicture}
            % Region Shading
            \pattern[pattern={Lines[angle=45,distance=0.1cm]}, pattern color=red]
                (0cm, 0cm) -- (0cm,  0.6cm) -- (-1.8cm,  0.6cm) -- (-1.8cm, 0cm) -- cycle;
            \pattern[pattern={Lines[angle=45,distance=0.1cm]}, pattern color=green]
                (0cm, 0cm) -- (0cm, -0.6cm) -- (-1.8cm, -0.6cm) -- (-1.8cm, 0cm) -- cycle;
            \pattern[pattern={Lines[angle=45,distance=0.1cm]}, pattern color=cyan]
                (0cm, 0cm) -- (0cm, -0.6cm) -- ( 1.8cm, -0.6cm) -- ( 1.8cm, 0cm) -- cycle;
            \pattern[pattern={Lines[angle=45,distance=0.1cm]}, pattern color=yellow]
                (0cm, 0cm) -- (0cm,  0.6cm) -- ( 1.8cm,  0.6cm) -- ( 1.8cm, 0cm) -- cycle;

            % Axes
            \draw[axis] (-2cm, 0cm) -- (2cm, 0cm) node[right] {$\Re$};
            \draw[axis] (0cm, -2cm) -- (0cm, 2cm) node[right] {$\Im$};
            \node (name) at (1.5cm, 1.5cm) {$z$};
            \draw (name.north west) |- (name.south east);
    
            % Conductors
            \draw[charged strip, blue] (0cm,  1.8cm) -- (1.8cm,  1.8cm);
            \draw[charged strip]       (0cm,  1.2cm) -- (1.8cm,  1.2cm);
            \draw[charged strip, blue] (0cm,  0.6cm) -- (1.8cm,  0.6cm);
            \draw[charged strip]       (0cm,    0cm) -- (1.8cm,    0cm);
            \draw[charged strip, blue] (0cm, -0.6cm) -- (1.8cm, -0.6cm);
            \draw[charged strip]       (0cm, -1.2cm) -- (1.8cm, -1.2cm);
            \draw[charged strip, blue] (0cm, -1.8cm) -- (1.8cm, -1.8cm);
            
            \node[right, orange] at (1.8cm, 1.2cm) {$+V_0$};
            \node[right, blue]   at (1.8cm, 0.6cm) {$-V_0$};

        \end{tikzpicture}
        \caption{}%
        \label{subfig:problem5-5:original}
    \end{subfigure}
    \hfill
    \begin{subfigure}{0.4\textwidth}
        \centering
        \begin{tikzpicture}
            % Region Shading
            \pattern[pattern={Lines[angle=45,distance=0.1cm]}, pattern color=red]
                (-0.6cm, 0cm) arc ( 180:0:0.6cm) -- cycle;
            \pattern[pattern={Lines[angle=45,distance=0.1cm]}, pattern color=green]
                (-0.6cm, 0cm) arc (-180:0:0.6cm) -- cycle;
            \pattern[pattern={Lines[angle=45,distance=0.1cm]}, pattern color=cyan]
                (1.8cm, -1.8cm) -- (1.8cm, 0cm) -- (0.6cm, 0cm) arc (0:-180:0.6cm)
                -- (-0.6cm, 0cm) -- (-1.8cm, 0cm) -- (-1.8cm, -1.8cm) -- cycle;
            \pattern[pattern={Lines[angle=45,distance=0.1cm]}, pattern color=yellow]
                (1.8cm,  1.8cm) -- (1.8cm, 0cm) -- (0.6cm, 0cm) arc (0: 180:0.6cm)
                -- (-0.6cm, 0cm) -- (-1.8cm, 0cm) -- (-1.8cm,  1.8cm) -- cycle;
            
            % Axes
            \draw[axis] (-2cm, 0cm) -- (2cm, 0cm) node[right] {$\Re$};
            \draw[axis] (0cm, -2cm) -- (0cm, 2cm) node[right] {$\Im$};
            \node (name) at (1.5cm, 1.5cm) {$W$};
            \draw (name.north west) |- (name.south east);
    
            % Conductors
            \draw[charged strip]       ( 0.6cm, 0cm) -- ( 1.8cm, 0cm);
            \draw[charged strip, blue] (-0.6cm, 0cm) -- (-1.8cm, 0cm);
            \draw[gray, dashed] (0.6cm, 0cm) arc (0:360:0.6cm);
            \node[right, gray] at (0.5cm, 0.5cm) {$r = 1$};

        \end{tikzpicture}
        \caption{}%
        \label{subfig:problem5-5:exp}
    \end{subfigure}
    \hfill
    \begin{subfigure}{0.4\textwidth}
        \centering
        \begin{tikzpicture}
            % Region Shading
            \pattern[pattern={Lines[angle=45,distance=0.1cm]}, pattern color=red]
                (-0.9425cm, 0cm)
                plot[domain=-1.57:1.57, smooth, variable=\x]
                    ({0.6*\x},{0.6*0.8813*(1-0.4053*\x*\x)})
                -- cycle;
            \pattern[pattern={Lines[angle=45,distance=0.1cm]}, pattern color=green]
                (-0.9425cm, 0cm)
                plot[domain=-1.57:1.57, smooth, variable=\x]
                    ({0.6*\x},{-0.6*0.8813*(1-0.4053*\x*\x)})
                -- cycle;
            \pattern[pattern={Lines[angle=45,distance=0.1cm]}, pattern color=cyan]
                (-0.9425cm, 0cm)
                plot[domain=-1.57:1.57, smooth, variable=\x]
                    ({0.6*\x},{-0.6*0.8813*(1-0.4053*\x*\x)})
                -- (0.9425cm, -1.8cm) -- (-0.9425cm, -1.8cm) -- cycle;
            \pattern[pattern={Lines[angle=45,distance=0.1cm]}, pattern color=yellow]
                (-0.9425cm, 0cm)
                plot[domain=-1.57:1.57, smooth, variable=\x]
                    ({0.6*\x},{0.6*0.8813*(1-0.4053*\x*\x)})
                -- (0.9425cm, 1.8cm) -- (-0.9425cm, 1.8cm) -- cycle;
            
            % Axes
            \draw[axis] (-2cm, 0cm) -- (2cm, 0cm) node[right] {$\Re$};
            \draw[axis] (0cm, -2cm) -- (0cm, 2cm) node[right] {$\Im$};
            \node (name) at (1.5cm, 1.5cm) {$\zeta$};
            \draw (name.north west) |- (name.south east);
    
            % Conductors
            \draw[charged strip]       ( 0.9425cm, -1.8cm) -- ( 0.9425cm, 1.8cm);
            \draw[charged strip, blue] (-0.9425cm, -1.8cm) -- (-0.9425cm, 1.8cm);
            \draw[gray, dashed, domain=-1.57:1.57, smooth, variable=\x]
                plot ({0.6*\x},{0.6*0.8813*(1-0.4053*\x*\x)});
            \draw[gray, dashed, domain=-1.57:1.57, smooth, variable=\x]
                plot ({0.6*\x},{-0.6*0.8813*(1-0.4053*\x*\x)});
            \node[right, yshift=-0.2cm] at ( 0.9425cm, 0cm) {$+V_0$};
            \node[ left, yshift=-0.2cm] at (-0.9425cm, 0cm) {$-V_0$};

        \end{tikzpicture}
        \caption{}%
        \label{subfig:problem5-5:final}
    \end{subfigure}
    \caption{Results of conformal mappings used in \refprob~5--5.}%
    \label{fig:problem5-5}
\end{figure}

\begin{enumerate}[wide, labelindent = 0pt, label = (\alph*)]
\item
\urlfoot{https://arxiv.org/pdf/2205.11916}{Let's think step by step.}
That is, let us consider the intermediary tranformation
\[
    W := e^{\frac{\pi z}{d}},
\]
which is plausible since it is periodic with period $id$.
Its effect is shown in \reffig~\ref{subfig:problem5-5:exp},
with matching regions correspondingly shaded.

Of course, a manual that claims to be useful should explain how this can be drawn by hand.
The orange conductor(s) is (are) mapped to the real interval $[1, \infty)$,
while the blue conductors to the interval $(-\infty, -1]$,
which is straightforward to see.
(Note that for the blue conductor,
\[
    e^{\frac{\pi}{d}(x + (2n+1)id)} = -e^{\frac{\pi x}{d}}
\]
would account for the negative sign.)
The gray boundary is the image of the imaginary interval $[-id, id]$
(if such a notation makes sense), which is given by
\[
    \left\{ e^{\frac{i\pi t}{d}} : t \in [-d, d] \right\}
\]
which traces out the unit circle.

We finally apply the arcsine transformation, which we have encountered in both
the example regarding \refeq{5}{9} in the main text and \refprob~5--1.
The result of this transformation is given in \reffig~\ref{subfig:problem5-5:final},
where the two conductors are now parallel plates, and we only consider the region between them.
At this point, how the regions are transformed doesn't really matter,
but they are pretty so they get to stay.\footnote{%
Note that the boundary is only approximately drawn;
the actual boundary is given by (ignoring scales)
$y = \pm \cosh^{-1}\left( \sqrt{1 + \cos^2 x} \right)$,
while I have drawn two parabolas with the same zeros and peaks.
They do look quite similar according to Desmos, but I felt like I should mention it.
}

The problem states that now the potential-finding problem is trivial.
I don't like the word `trivial,' but it seems safe to say that this problem is way easier now.
We need to find an analytic function $F(\zeta)$ such that
\[
    \evalAt{\Im F(\zeta)}{\Re \zeta = \pm V_0} = \pm V_0.
\]
This can be achieved by the function $F(\zeta) := i\zeta$ (quite simple indeed!).
\[
    \Rightarrow F(\zeta(z))
    = \frac{2iV_0}{\pi} \sin^{-1} e^{\frac{\pi z}{d}}
\]
\begin{align*}
    \therefore V(x, y)
    &= \Im F(\zeta(z)) \\
    &= \frac{2V_0}{\pi} \Re\left\{ \sin^{-1} e^{\frac{\pi z}{d}} \right\}
\end{align*}

\item
We use the stream function to calculate the charge accumulated on each plate.
By symmetry, we only need to calculate the charge density on $x + i0^+$.
\begin{align*}
    \Re F(x + i0^+)
    &= \frac{2V_0}{\pi} \Im\left\{ \sin^{-1} e^{\frac{\pi x}{d} + i0^+} \right\} \\
    &= \frac{2V_0}{\pi} \Im\left\{
        \halfpi + i\cosh^{-1} \left( e^{\frac{\pi x}{d}} \right)
    \right\} \\
    &= \frac{2V_0}{\pi} \ln \left( e^{\frac{\pi x}{d}} + \sqrt{e^{\frac{2\pi x}{d}} - 1} \right)
\end{align*}

Thus, the argument of \refprob~5--4 then tells us that
\[
    \lambda(0 + i0^+, l + i0^+)
    = \frac{2\epsilon_0 V_0}{\pi} \ln \left( e^{\frac{\pi l}{d}} + \sqrt{e^{\frac{2\pi l}{d}} - 1} \right)
\]
where $\lambda$ denotes the charge per unit length perpendicular to the $z$ plane.
We should note that this is the total charge accumulated,
which is equally divided between the top and bottom of each conductor.
Thus, the capacitance per unit length is given by
\begin{align*}
    C
    &= \frac{\lambda / 2}{V_0} \\
    &= \frac{\epsilon_0}{\pi} \ln \left(
        e^{\frac{\pi l}{d}} + \sqrt{e^{\frac{2\pi l}{d}} - 1}
    \right) \\
    &\approx \frac{\epsilon_0}{\pi} \ln \left( 2e^{\frac{\pi l}{d}} \right) \\
    &= \frac{\epsilon_0}{d} \left( l + \frac{d}{\pi}\ln 2 \right)
\end{align*}
where the approximation holds for $l >> d$ (i.e., semi-infinite plates).
Identifying $\frac{\epsilon_0 l}{d}$ as the capacitance of infinite-plate capacitors,
we conclude that the additional term $\frac{d}{\pi}\ln 2$ must come from edge effects.

\end{enumerate}
