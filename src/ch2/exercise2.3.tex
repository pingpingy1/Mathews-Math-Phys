%%%%%%%%%%%%%%%%%%%%%%%
\item
%%%%%%%%%%%%%%%%%%%%%%%

To describe the sign pattern, we shall use
\[
	\left\{ \Re \left\{ e^{i \left( \frac{n}{3} + \frac{1}{6} \right)} \pi \right\} \right\}
	= \frac{\sqrt{3}}{2},\; 0,\; -\frac{\sqrt{3}}{2},\; -\frac{\sqrt{3}}{2},\; 0,\; \frac{\sqrt{3}}{2},\; \cdots.
\]
\begin{align*}
	(\text{Given series})
	&= \frac{1}{1 \cdot 3^0} + \frac{0}{3 \cdot 3^1} + \frac{-1}{5 \cdot 3^2} + \frac{-1}{7 \cdot 3^3} + \frac{0}{9 \cdot 3^4} + \frac{1}{11 \cdot 3^5} + \cdots \\
	&= \sum_{n = 0}^\infty \frac{1}{(2n + 1) 3^n} \cdot \frac{2}{\sqrt{3}} \Re \left\{ e^{i \left( \frac{n}{3} + \frac{1}{6} \right)} \pi \right\} \\
	&= \Re \left\{ \frac{2}{\sqrt{3}} e^{i \frac{\pi}{6}} \sum_{n = 0}^\infty \frac{1}{2n + 1} {\left( \frac{1}{3} e^{i \frac{\pi}{3}} \right)}^n \right\} \\
	&= \Re \left\{ 2 \sum_{n = 0}^\infty \frac{1}{2n + 1} {\left( \frac{1}{\sqrt{3}} e^{i \frac{\pi}{6}} \right)}^{2n + 1} \right\} \\
	&= \Re \left\{ 2 \tanh^{-1} \left( \frac{1}{\sqrt{3}} e^{i\frac{\pi}{6}} \right) \right\} \\
	&= \Re \left\{ \ln \left( \frac{1 + \frac{1}{\sqrt{3}} e^{i\frac{\pi}{6}}}{1 - \frac{1}{\sqrt{3}} e^{i\frac{\pi}{6}}} \right) \right\} \\
	&= \ln \left| \frac{1 + \frac{1}{\sqrt{3}} e^{i\frac{\pi}{6}}}{1 - \frac{1}{\sqrt{3}} e^{i\frac{\pi}{6}}} \right| \\
	&= \ln \left| \frac{5 - \sqrt{3} i}{2} \right| \\
	&= \half \ln 7
\end{align*}
