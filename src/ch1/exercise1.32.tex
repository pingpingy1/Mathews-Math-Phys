%%%%%%%%%%%%%%%%%%%%%%%
\item
%%%%%%%%%%%%%%%%%%%%%%%
Recall that
\[
	\frac{c_{r + 2}}{c_r} = \frac{(r + m - n)(r + m + n + 1)}{(r + 1)(r + 2)}
	\text{ and } v(x) = \sum_{r=0}^\infty c_r x^r
\]
where the fraction is well-defined since we are assuming that $\{c_r\}$ never terminates.
Notice how
\[
	\frac{c_{r + 2}}{c_r} \approx \frac{r + 2m  + 1}{r + 3}
\]
for large values of $r$.

On the other hand, the definition of binomial coefficients
\[
	\binom{-m}{r} := \frac{-m \cdot (-m - 1) \cdot \cdots \cdot (-m - r + 1)}{r!}
\]
naturally yields
\begin{align*}
	\frac{\binom{-m}{r + 2}}{\binom{-m}{r}}
	&= \frac{(-m - r)(-m - r - 1)}{(r + 1)(r + 2)} \\
	&= \frac{(r + m)(r + m + 1)}{(r + 1)(r + 2)} \\
	&\approx \frac{r + 2m + 1}{r + 3}
\end{align*}
for large values of $r$.

Therefore, $c_r$ behaves like $\binom{-m}{r}$ as $r$ grows without bound, and consequently,
\[
	v(x) \approx {\left( 1 - x^2 \right)}^{-m}.
\]