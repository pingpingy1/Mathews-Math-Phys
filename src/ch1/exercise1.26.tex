%%%%%%%%%%%%%%%%%%%%%%%
\item
%%%%%%%%%%%%%%%%%%%%%%%
While the use of Dirac delta ``function'' must be dealt with caution,
we are physicists so we shall simply accept that they exist.
\begin{align*}
	f(x)
	= y'' + py' + q
	&= \int_a^b dx' f(x') \left( \partialsecondderiv{x} + p(x) \partialderiv{x} + q(x) \right) G(x, x') \\
	&= \int_a^b dx' f(x') \delta(x - x')
\end{align*}
\[
	\Rightarrow \left( \partialsecondderiv{x} + p(x) \partialderiv{x} + q(x) \right) G(x, x') = \delta(x - x')
\]
From this, we see that $G(x, x')$ must separately equal zero for both $x < x'$ and $x > x'$.
We thus write
\[
	G(x, x') =
	\begin{cases}
		y_1(x) \alpha(x') + y_2(x) \beta(x') & (x < x') \\
		y_1(x) \gamma(x') + y_2(x) \zeta(x') & (x > x')
	\end{cases}.
\]
From the definition of delta functions, we find that:
\begin{align*}
	1
	&= \int_{x' - \epsilon}^{x' + \epsilon} dx \delta(x - x') \\
	&= \int_{x' - \epsilon}^{x' + \epsilon} dx \left( \partialsecondderiv{x} + p(x) \partialderiv{x} + q(x) \right) G(x, x') \\
	&\approx {\left[ \left( \partialderiv{x} + p(x) \right) G(x, x') \right]}_{x' - \epsilon}^{x' + \epsilon}
		     - \int_{x' - \epsilon}^{x' + \epsilon} dx p'(x) G(x, x') \\
	&\approx {\left[ \partialderiv{x} G(x, x') \right]}_{x' - \epsilon}^{x' + \epsilon} \\
	&\approx \lim_{x \rightarrow x'+} \partialderiv{x} G(x, x') - \lim_{x \rightarrow x'-} \partialderiv{x} G(x, x')
\end{align*}
where we take infinitesimally small $\epsilon$.

The boundary conditions for $y$ imply
\[
	y(a) = y(b) = 0
\]
\[
	\Rightarrow G(a, x') = G(b, x') = 0
\]
\[
	\Rightarrow y_1(a)\alpha(x') + y_2(a)\beta(x') = y_1(b)\alpha(x') + y_2(b)\beta(x') = 0
\]
\[
	\Rightarrow \beta(x') = \gamma(x') = 0.
\]
The derivative condition, in turn, yields
\[
	y_2'(x')\zeta(x') - y_1'(x')\alpha(x') = 1.
\]
As any such choice of $\zeta(x')$ and $\alpha(x')$ leads to a valid Green's function,
we may arbitrarily choose
\[
	\alpha(x') = y_2'(x'),\; \zeta(x') = y_1'(x') + \frac{1}{y_2'(x')}
\]
\[
	\Rightarrow G(x, x') =
	\begin{cases}
		y_1(x)y_2'(x') & (x < x') \\
		y_2(x)y_1'(x') + \frac{y_2(x)}{y_2'(x')} & (x > x')
	\end{cases}.
\]
\[
	\therefore y(x)
	= y_1(x) \int_x^b dx' f(x') y_2'(x') + y_2(x) \int_a^x f(x') \left( y_1'(x') + \frac{1}{y_2'(x')} \right)
\]

For the given example $y'' + k^2 y = f(x)$, we may take
\[
	y_1(x) = \sin k(x - a),\; y_2(x) = \sin k(b - x)
\]
provided that $b - a$ is not an integer multiple of the period $\frac{2\pi}{k}$.
This leads to
\[
	G(x, x') = 
	\begin{cases}
		-k\sin k(x - a) \cos k(b - x) & (x < x') \\
		\sin k(b - x) \left( k\cos k(x' - a) - \frac{1}{k \cos k (b - x')} \right) & (x > x')
	\end{cases}
\]
\begin{align*}
	\Rightarrow y
	= &-k\sin k(x - a) \int_x^b dx' f(x') \cos k(b - x')\\
	  &+ \sin k(b - x) \int_a^x dx' f(x') \left( k\cos k(x' - a) - \frac{1}{k \cos k (b - x')} \right).
\end{align*}
One may verify that this solution satisfies both the boundary conditions and the given differential equation.