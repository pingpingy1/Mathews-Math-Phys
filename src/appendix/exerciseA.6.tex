%%%%%%%%%%%%%%%%%%%%%%%
\item[6.] % TODO: Solve A--1 to A--5 and remove this!
%%%%%%%%%%%%%%%%%%%%%%%

(This single problem had cost me months of my life, and here's the unsatisfying conclusion of that journey.
I am fairly certain that there are a myriad of different possible answers for this problem with varying levels of sophistication,
and possibly even more ways to derive said solutions.
For my purposes, I have settled on a solution that is semi-straightforward with an answer that I can plug into almighty WolframAlpha.)

We will derive a procedure to (painstakingly) calculate the Laurent series of the given function.
First, let us define some notations.
Recall that if a function $f(z)$ is analytic on an annulus around a point $z_0$,
then it can be expanded into a unique Laurent series
\begin{align*}
    f(z)
    &= \sum_{k=-\infty}^\infty a_k {(z - z_0)}^k \\
    &= \cdots + \frac{a_{-2}}{{(z - z_0)}^2} + \frac{a_{-1}}{z - z_0}
     + a_0 + a_1 (z - z_0) + a_2 {(z - z_0)}^2 + \cdots.
\end{align*}
For Laurent series specifically centered around the origin,
let us denote the Laurent coefficients as
\[
    \Laurentcoeff{k}{f(z)} := a_k.
\]
Then, multiplying a monomial corresponds to
\[
    \Laurentcoeff{k}{z^l f(z)} = \Laurentcoeff{k - l}{f(z)}.
\]
Also, the residue at the origin can be compactly expressed as
\[
    \res{z=0}{f(z)} = \Laurentcoeff{-1}{f(z)}.
\]

The residue asked of us is given by
\begin{align*}
    \res{z=\pi}{z^2 e^{\inv{\sin z}}}
    &= \res{z=0}{{(z + \pi)}^2 e^{\inv{\sin (z + \pi)}}} \\
    &= \Laurentcoeff{-1}{\left( z^2 + 2\pi z + \pi^2 \right) e^{-\inv{\sin z}}} \\
    &= \Laurentcoeff{-3}{e^{-\inv{\sin z}}}
     + 2\pi \Laurentcoeff{-2}{e^{-\inv{\sin z}}}
     + \pi^2 \Laurentcoeff{-1}{e^{-\inv{\sin z}}}.
\end{align*}
Thus, our job is reduced to calculating the Laurent coefficients of $e^{-\inv{\sin z}}$.
We expand the exponential into its Taylor series and obtain
\[
    \Laurentcoeff{-n}{e^{-\inv{\sin z}}}
    = \sum_{k=0}^\infty \frac{{(-1)}^k}{k!} \Laurentcoeff{-n}{{(\sin z)}^{-k}}.
\]

Squaring, cubing, and so forth a power series (or even a Laurent series) is something achievable,
but inverting a power series is a much more complicated task.
We thus investigate a general method of calculating the inverse powers of a general series.
Consider the following manipulation:
\begin{align*}
    \Laurentcoeff{-n}{{(f(z))}^{-k}}
    &= \Laurentcoeff{-1}{z^{n - 1} {(f(z))}^{-k}} \\
    &= \Laurentcoeff{-1}{{\left( \frac{z^n}{n} \right)}' {(f(z))}^{-k}} \\
    &= \Laurentcoeff{-1}{{\left( \frac{z^n}{n} {(f(z))}^{-k} \right)}'}
     - \Laurentcoeff{-1}{\frac{z^n}{n} {\left( {(f(z))}^{-k} \right)}'}
\end{align*}
which comes mostly from playing around, but perhaps reasonable since the $-1$th%
\footnote{I had to search for whether it should be `st' or `th'; see \url{https://english.stackexchange.com/questions/326604/is-it-correct-to-say-1th-or-1st}.}
coefficient seems to be special in certain ways.
In fact, it is so special that we can immediately see that the first term must be zero,
regardless of whether $f(z)$ is analytic at the origin!
This is because no derivative of a monomial yields the $-1$th power term.%
\footnote{This can also be seen via $\res{z=0}{f(z)} = \inv{2i\pi} \oint dz f'(z) = 0$ around the origin.}
Thus,
\begin{align*}
    \Laurentcoeff{-n}{{(f(z))}^{-k}}
    &= -\Laurentcoeff{-1}{\frac{z^n}{n} {\left( {(f(z))}^{-k} \right)}'} \\
    &= \frac{k}{n} \Laurentcoeff{-1}{z^n {(f(z))}^{-k-1} f'(z)}.
\end{align*}
Let $g(z) := z^n {(f(z))}^{-k-1} f'(z)$ and $g_{-1} := \Laurentcoeff{-1}{g(z)}$.
The Laurent series of $g(z) - g_{-1} z^{-1}$ contains no $z^{-1}$ term,
and thus can be integrated to an analytic function; let us call it $G(z)$.
Hence, $g(z) = G'(z) + g_{-1} z^{-1}$.
\[
    g(f^{-1}(z)) {\left( f^{-1} (z) \right)}'
    = {\left( G(f^{-1} (z)) \right)}'+ g_{-1} \frac{{\left( f^{-1} (z) \right)}'}{f^{-1} (z)}
\]
Again, ${\left( G(f^{-1} (z)) \right)}'$ has zero residue.
If the residue of $\frac{{\left( f^{-1} (z) \right)}'}{f^{-1} (z)}$ could be easily calculated and is nonzero,
then we get an elegant formula for $g_{-1}$ as
\begin{align*}
    g_{-1}
    &= \frac{
        \Laurentcoeff{-1}{g(f^{-1}(z)) {\left( f^{-1} (z) \right)}'}
    }{
        \Laurentcoeff{-1}{\frac{{\left( f^{-1} (z) \right)}'}{f^{-1} (z)}}
    } \\
    &= \frac{
        \Laurentcoeff{-1}{{(f^{-1} (z))}^n z^{-k-1} f'\left( f^{-1} (z) \right) {\left( f^{-1} (z) \right)}'}
    }{
        \Laurentcoeff{-1}{\frac{{\left( f^{-1} (z) \right)}'}{f^{-1} (z)}}
    } \\
    &= \frac{\Laurentcoeff{k}{{(f^{-1} (z))}^n}}{
        \Laurentcoeff{-1}{\frac{{\left( f^{-1} (z) \right)}'}{f^{-1} (z)}}
    }
\end{align*}
and consequently
\[
    \Laurentcoeff{-n}{{(f(z))}^{-k}}
    = \frac{k}{n} \frac{\Laurentcoeff{k}{{(f^{-1} (z))}^n}}{
        \Laurentcoeff{-1}{\frac{{\left( f^{-1} (z) \right)}'}{f^{-1} (z)}}
    }.
\]
Now, \emph{if} $f(z)$ is analytic at the origin \emph{and} $f(0) = 0$ \emph{and} $f'(0) \neq 0$
(which just so happens to hold for $f(z) = \sin z$),
then one could show that these same conditions hold for $f^{-1} (z)$ and consequently
\[
    \Laurentcoeff{-1}{\frac{{\left( f^{-1} (z) \right)}'}{f^{-1} (z)}} = 1.
\]
(This could be shown by substituting $n = -1$ for \refprob~A--4,
since we did not require that $n$ be positive anyway!)
Therefore, we arrive at the pivotal formula
\[
    \Laurentcoeff{-n}{{(f(z))}^{-k}} = \frac{k}{n} \Laurentcoeff{k}{{(f^{-1} (z))}^n}.
\]
This is known in the maths world as the
\urlfoot{https://en.wikipedia.org/wiki/Formal_power_series\#The_Lagrange_inversion_formula}{Lagrange inversion formula},
and it is what we desperately needed to calculate the residue in question!

Having sufficiently patted ourselves on our respective backs after that long derivation,
let us go back to the original question.
We have essentially derived the formula
\[
    \Laurentcoeff{-n}{{(\sin z)}^{-k}} = \frac{k}{n} \Laurentcoeff{k}{{(\arcsin z)}^n}.
\]
(Note that I use $\arcsin$ in lieu of $\sin^{-1}$ as all these exponents are getting confusing.)
Let us substitute this formula into the previous equation:
\begin{align*}
    \Laurentcoeff{-n}{e^{-\inv{\sin z}}}
    &= \sum_{k=0}^\infty \frac{{(-1)}^k}{k!} \Laurentcoeff{-n}{{(\sin z)}^{-k}} \\
    &= \sum_{k=0}^\infty \frac{{(-1)}^k}{k!} \cdot \frac{k}{n} \Laurentcoeff{k}{{(\arcsin z)}^n} \\
    &= \inv{n} \sum_{k=1}^\infty \frac{{(-1)}^k}{(k-1)!} \Laurentcoeff{k}{{(\arcsin z)}^n}.
\end{align*}
As for the powers of $\arcsin z$, let us simply expand out the multiplications
using the Taylor series of $\arcsin z$ only.
\begin{align*}
    \arcsin z
    &= \int_0^z \frac{du}{\sqrt{1 - u^2}} \\
    &= \int_0^z dt \sum_{t=0}^\infty \binom{-\half}{t} u^{2t} \\
    &= \sum_{t=0}^\infty \frac{(-1/2) (-3/2) \cdots (1/2 - t)}{2^t t!} \cdot \frac{u^{2t + 1}}{2t + 1} \\
    &= \sum_{t=0}^\infty \frac{{(-1)}^t (2t)!}{4^t (2t+1) {(t!)}^2} u^{2t + 1}
\end{align*}
\begin{align*}
    \Rightarrow \Laurentcoeff{k}{{(\arcsin z)}^n}
    &= \sum_{t_1 + \cdots + t_k = n} \prod_{j=1}^n \Laurentcoeff{t_j}{\arcsin z} \\
    &= \sum_{(2l_1 + 1) + \cdots + (2l_k + 1) = n} \prod_j \Laurentcoeff{2l_j + 1}{\arcsin z} \\
    &= \sum_{2\sum_j l_j + k = n} {\left( -\inv{4} \right)}^{\sum_j l_j}
       \prod_j \left( \frac{(2l_j)!}{(2l_j + 1) {(l_j !)}^2} \right)
\end{align*}
\begin{align*}
    \Rightarrow \Laurentcoeff{-n}{e^{-\inv{\sin z}}}
    &= \inv{n} \sum_{k=1}^\infty \frac{{(-1)}^k}{(k-1)!} \Laurentcoeff{k}{{(\arcsin z)}^n} \\
    &= \inv{n} \sum_{k=1}^\infty \frac{{(-1)}^k}{(k-1)!} \\
    &\quad \times \sum_{2\sum_j l_j + k = n} {\left( -\inv{4} \right)}^{\sum_j l_j}
            \prod_j \left( \frac{(2l_j)!}{(2l_j + 1) {(l_j !)}^2} \right) \\
    &= \inv{n} \sum_{l_1=0}^\infty \cdots \sum_{l_n=0}^\infty
       \frac{{(-1)}^{n - 2\sum_j l_j}}{\left( n - 1 + 2\sum_j l_j \right)!} \\
    &\quad \times {\left( -\inv{4} \right)}^{\sum_j l_j}
            \prod_j \left( \frac{(2l_j)!}{(2l_j + 1) {(l_j !)}^2} \right) \\
    &= \inv{n} \sum_{l_1=0}^\infty \cdots \sum_{l_n=0}^\infty
       \frac{{(-1)}^{\sum_j l_j + n}}{4^{\sum_j l_j} \left( n - 1 + 2\sum_j l_j \right)!} \\
    &\quad \times \prod_j \left( \frac{(2l_j)!}{(2l_j + 1) {(l_j !)}^2} \right)
\end{align*}

We evaluate this formula for $n = 1, 2, 3$.
\[
    \Laurentcoeff{-1}{e^{-\inv{\sin z}}}
    = \sum_{l_1 = 0}^\infty \frac{{(-1)}^{l_1 + 1}}{4^{l_1} (2l_1 + 1) {(l_1 !)}^2}
\]
\[
    \Laurentcoeff{-2}{e^{-\inv{\sin z}}}
    = \half \sum_{l_1 = 0}^\infty \sum_{l_2 = 0}^\infty
      \frac{{(-1)}^{l_1 + l_2}}{4^{l_1 + l_2} (2l_1 + 2l_2 + 1)!}
      \frac{(2l_1)!}{(2l_1 + 1) {(l_1 !)}^2} \frac{(2l_2)!}{(2l_2 + 1) {(l_2 !)}^2}
\]
\begin{align*}
    \Laurentcoeff{-3}{e^{-\inv{\sin z}}}
    &= \inv{3} \sum_{l_1 = 0}^\infty \sum_{l_2 = 0}^\infty \sum_{l_3 = 0}^\infty
       \frac{{(-1)}^{l_1 + l_2 + l_3 + 1}}{4^{l_1 + l_2 + l_3} (2l_1 + 2l_2 + 2l_3 + 2)!} \\
    &\quad \times \frac{(2l_1)!}{(2l_1 + 1) {(l_1 !)}^2} \frac{(2l_2)!}{(2l_2 + 1) {(l_2 !)}^2}
            \frac{(2l_3)!}{(2l_3 + 1) {(l_3 !)}^2}
\end{align*}

\[
\begin{split}
    \therefore \res{z=\pi}{z^2 e^{\inv{\sin z}}}
    &= \sum_{l_1=0}^\infty \frac{{(-1)}^{l_1 + 1} (2l_1)!}{4^{l_1} (2l_1 + 1) {(l_1 !)}^2} \left(
        \frac{\pi^2}{(2l_1)!}
    \right.\\
    &\quad - \sum_{l_2=0}^\infty \frac{{(-1)^{l_2}} (2l_2)!}{4^{l_2} (2l_2 + 1) {(l_2 !)}^2} \left(
        \frac{\pi}{(2l_1 + 2l_2 + 1)!}
    \right. \\
    &\qquad - \left.\left. \sum_{l_3=0}^\infty \frac{{(-1)^{l_3}} (2l_3)!}{4^{l_3} (2l_3 + 1) {(l_3 !)}^2}
       \cdot \inv{3(2l_1 + 2l_2 + 2l_3 + 2)!} \right)\right)
\end{split}
\]

\paragraph{Concluding remarks:}
\begin{enumerate}[wide, labelindent = 0pt, label = (\roman*)]
\item
This result, while hard achieved on my part, is still pretty shitty;
for one, it cannot be calculated via WolframAlpha alone.%
\footnote{If any Mathematica wizards could numerically calculate this, I'm all ears.}
Some points for improvement include:
\begin{itemize}
    \item More compact expressions for the Taylor coefficients of ${(\arcsin z)}^k$
    \item A numerical evaluation of the residue
    \item Even a closed-form solution?
    (This sounds crazy, but $\Laurentcoeff{-1}{e^{-\inv{\sin z}}}$ \emph{does} have a closed-form expression!
    It uses a hypergeometric function, somehow.)
\end{itemize}
And we all know that ``We leave so-and-so to future work.'' translates to
either ``I give up.'' or ``Give me more money.''

\item
I have been hinted at that this problem could also be solved via a contour integral method,
where we try to evaluate
\[
    \oint_{\abs{z} = 1} dz z^2 e^{\inv{\sin z}}
\]
using some tabulated-function trickery.
I'm too tired to pursue this path, but it is still another viable method.

\end{enumerate}